\documentclass{article}
\usepackage{graphicx} % Required for inserting images
\usepackage{sidecap}
\usepackage{wrapfig,lipsum}
\usepackage[utf8]{inputenc} %lettere accentate da tastiera 
\usepackage[T1]{fontenc} % higher quality font encoding
%load the font and set it to default
\usepackage{amsmath,amsthm,amsfonts,amssymb}
\numberwithin{equation}{section}
\usepackage[english,italian]{babel}
\usepackage{url}
\usepackage{geometry}
\geometry{a4paper,top=3cm,bottom=3cm,left=3.5cm,right=3.5cm,%
	heightrounded,bindingoffset=5mm}
\usepackage[x11names]{xcolor}
\usepackage[most]{tcolorbox}
\tcbuselibrary{theorems}
\usepackage{physics}
\usepackage{hyperref}
\hypersetup{
	colorlinks=true,
	linkcolor=pink,
	filecolor=magenta,      
	urlcolor=blue,
	pdftitle={Analisi II},
	pdfpagemode=FullScreen,
}
\usepackage{enumitem}
\usepackage[toc,page]{appendix}
\usepackage{biblatex}

\newtheorem{teorema}{Teorema}[subsection]

\theoremstyle{definition}
\newtheorem{definizione}{Definizione}[section]

\newtheorem*{proprieta}{Proprietà}
\newtheorem{corollario}{Corollario}[teorema]
\newtheorem*{formula}{Formula}
\newtheorem{prop}{Proposizione}[section]
\newtheorem{es}{Esempio}[section]
\newtheorem{lemma}{Lemma}[section]
\newtcbtheorem[number within=section]{teo}{Teorema}{colback=LightCyan1!40 ,colframe=RoyalBlue1!100,sharp corners,separator sign dash,fonttitle=\bfseries}{thm}
\newtcbtheorem[number within=section]{teo1}{}{colback=black!5 ,colframe=Burlywood4!80 }{thm}



\newcommand{\R}{\mathbb{R}}
\newcommand{\D}{\mathbb{D}}
\newcommand{\V}{\mathbb{V}}
\newcommand{\K}{\mathbb{K}}
\newcommand{\w}{\mathbb{W}}
\newcommand{\C}{\mathbb{C}}
\newcommand{\la}{\lambda}
\newcommand{\on}{^{\perp}}
\newcommand{\A}{\mathbb{A}}
\newcommand{\xb}{\overline{x}}
\newcommand{\fn}{f: A\subseteq \Rn \rightarrow \R}
\newcommand{\fnn}{f: A\subseteq \Rn \rightarrow \Rn}
\newcommand{\fnm}{f: A\subseteq \Rn \rightarrow \R^m}
\renewcommand{\labelitemi}{$\star$}
\renewcommand{\arraystretch}{1.5} % Aumenta l'altezza delle righe

\addbibresource{bibliografia.bib}
\title{Tirocinio}
\author{Mombelli Luca}


\begin{document}
	\maketitle
	\tableofcontents
	\newpage
	\section{Teorema di Nagumo}
	Utilizzerò la formulazione del teorema di Nagumo presentata nel libro "Viability Theory"\cite{VT}. Diamo ora alcune definizione necessarie 
	\begin{definizione}
		Sia K un sottoinsieme di uno spazio vettoriale finito dimensionale (oppure di uno spazio normato) X . Diciamo che una funzione $x (\cdot) : [0,T] \rightarrow X$ è \textit{viable} in K su $[0,T]$ se 
		$$\forall t \in [0,T] \ , \ \ x(t) \in K$$ 
	\end{definizione}
Consideriamo il seguente problema di cauchy 
\begin{equation}
	\label{diff:1}
\begin{cases}
	\dot{x}(t)= f(x(t)) \ \ \ \forall t \in [0,T ] \\ 
	x(0) = x_0
\end{cases}
\end{equation}
 con $f : \Omega \subset_{\text{op}} X \rightarrow X$ 
	\begin{definizione}
		Sia un sottoinsieme di $\Omega$. 
		Diciamo che K è\textit{ localmente viable} sotto $f$ se per ogni condizione iniziale $x_0 \in K $ , esiste un $T > 0$ e una soluzione viable su $[0,T]$  per l'equazione differenziale \ref{diff:1} con condizione iniziale $x_0$\\
		K è (globalmente) viable sotto f se possiamo sempre prendere $T= \infty$ 
	\end{definizione}
\begin{definizione}[Cono Tangente di Bouligand]
Sia W uno spazio normato , K un sottoinsieme non vuoto di W e sia x un elemento di K . Il cono tangente a K in x è l'insieme
$$T_K(x)= \{v \in W \ | \ \liminf_{h\rightarrow 0^+ }\frac{d_K(x+hv)}{h}=0\}$$
con $d_k(x):= \inf_{y \in K } \norm{x-y}$
\end{definizione}
Una definizione alternativa utilizza le successioni : \\
v appartiene a $T_K(x)$ se e solo se esiste una successione $h_n >0$ $h_n \rightarrow 0^+$ e una successione $v_n \in K$ , $v_n \rightarrow v $ tale che 
$$\forall n \in \mathbb{N} \ , \ x+h_nv_n \in K $$
È utile notare che se x è un punto interno al sottoinsieme K allora $T_k(x)=X$ . Nel caso in cui K sia aperto il cono tangente ad un qualsiasi punto di K è tutto lo spazio normato V.\\
\begin{lemma}
Sia $x: [0,T] \rightarrow K $ una funzione differenziabile e viable , allora 
$$\forall t \in [0,T) \ \ x'(t) \in T_K(x)$$
\end{lemma}
	\begin{definizione}[Viability Domain]
		Sia  k un sottoinsieme di  $\Omega$. Diciamo che k è  un viability domain della mappa $f : \Omega \rightarrow X$ se 
		$$\forall x \in X  \ \ , \ \ f(x) \in T_K(x)$$
	\end{definizione}
	
	\begin{teo}{Nagumo}{}
	 Suppiamo che il sottoinsieme K sia localmente compatto e che la funzione $f : K \rightarrow X$ sia continua . \\ Allora K è \textit{ localmente viable  } se e solo K è un viability domain
	\end{teo}
\begin{teo}{Viability}{}
	Consideriamo un sottoinsieme K di uno spazio finito dimensionale X e una mappa continua  $f : K \rightarrow X$. \\ Se k è un viability domain , allora per ogni condizione iniziale $x_0\in K $ esiste un T positivo e una soluzione viable su $[0,T]$ per l'equazione differenziale \ref{diff:1} con C.I $x_0$ tale che 
	$$\begin{cases}
		T= + \infty  \\ 
		T < +\infty \ \ \ \  e \ \ \ \  \limsup_{t\rightarrow T_-} \norm{x(t)}= \infty 
	\end{cases}$$
\end{teo}





%

\section{Richiami di Topologia}
\begin{definizione}[Spazio topologico] Sia X un insieme , una \textit{topoogia} su X , è    una famiglia $\tau$ di sottoinsiemi di X (i suoi elementi sono gli aperti di X ) che soddisfa le seguenti condizioni. 
	\begin{itemize}
		\item $\emptyset \ e \ X \in\tau$
		\item Unione arbitraria di aperti è un sottoinsieme aperto (se $A_{\lambda} \in \tau$ per ogni $\lambda \in \Lambda$ , allora $\bigcup_{\lambda \in \Lambda} A_{\lambda} \in \tau$)
		\item Intersezione finita di  aperti è un sottoinsieme aperto  (Se $A_1 , \dots , A_m \in \tau $ allora $A_1 \cap \dots \cap A_m \in \tau$0-)
	\end{itemize}
	Un insieme dotato di una topologia viene detto \textbf{spazio topologico}
\end{definizione}
\begin{es}
	Su ogni insieme X , $\tau = \{\emptyset , X\}$ è una topologia detta banale od indiscreta. \\
	Sull'insieme $\R$ 
	$$\tau_i = \{\emptyset , \R\} \cup \{\ ]a , + \infty [ \ \}$$
	è una topologia , topologia inferiore , similmente 
		$$\tau_{\sigma} = \{\emptyset , \R\} \cup \{\ ]-\infty  , a [\  \}$$ e1 la topologia superiore di $\R$
\end{es}
Una descrizione esplicita di tutti gli aperti di uno spazio topologico e1 impossibile , la topologia viene in genere descritta assegnando una base per essa 
\begin{definizione}	
	Sia $\tau$ una topologia su insieme X . una sottofamiglia (un insieme ) $B \subset \tau$ si dice una base di $\tau$ se ogni aperto $A \in \tau$ può essere scritt ocome unione di elementi d iB 	
	\end{definizione}
	\begin{teorema}
		Sia X un insieme  e $\mathcal{B} \subset P(X)$ una famiglia di suoi sottoinsiemi . Allora esiste una topologia su X di cui $\mathcal{B}$ è una base se e soltanto se sono soddisfatte le seguenti due condizioni 
		\begin{itemize}
			\item $X = \cup \{B \ |\  B \in \mathcal{B} \}$
			\item Per ogni coppia $A,B \in \mathcal{B}$ e per ogni punto $x \in A \cap B $ esiste $C \in \mathcal{B}$ tale che $x \in X \subset A \cup B$
		\end{itemize}
			\end{teorema}
		\subsection{Parte interna , chiusura ed intorni }
		\begin{definizione}
			Sia X uno spazio topologico e $B \subseteq X $. Si denota con 
			\begin{itemize}
\item $B^0$ l'unione di tutti gli aperti contenuti in B 
\item $\overline{B}$ l'intersezione di tutti i chiusi contenenti in B 
\item $\partial B = \overline{B} - B^0$
			\end{itemize}
			L'insieme $B^0$ viene detto parte interna di B ed è il più gtande aperto contenuto in B \\
			L'insieme $\overline{B}$ è il più piccolo chiuso contente B e viene detto chiusura di B \\
			Il sottoinsieme $\partial B $ è l'intersezione dei due chiusi $\overline{B}$ e $X - B^0$ e viene detto \textbf{frontiera} di B \\
		Osserviamo che un sottoinsieme B è aperto se e solo se $B=B^0$ e chiuse se $B= \overline{B}$
		\end{definizione}
		
\begin{definizione}
	Sia X uno spazio topologico e $x \in X$. Un sottoinsieme $U \subset X$ si dice \textit{intorno di x} se x è un punto interno di U , cioè se esiste un aperto V tale che $x\in V$ e $V \subset U $
\end{definizione}
Indichiamo con $\mathcal{I}(x) $ la famiglia di tutti gli intorni di x . per definizione se  A è un sottoinsieme di uno spazio topologico , allora $A^0 = \{ x \in A | A \in \mathcal{I}(x)\} $
\begin{definizione}
	Sia x un punto di uno spazio topologico X. Un sottofamiglia $ \mathcal{J}\subset \mathcal{I}(x)$ si dice \textit{base locale} oppure un sistema fondamentale di intorni di x , se per ogni $U \in \mathcal{I}(x)$ esiste $A \in \mathcal{J} $ tale che $A \subset U $
\end{definizione}
\begin{es}
	Sia $U \in \mathcal{I}(x)$ un intorno fissato . Allora tutti gli intorni di x contenuti in U formano un sistema fondamentale di intorni di x 
\end{es}
\subsection{Spazi Metrici}
\begin{definizione}
	Una distanza si di un insieme X è un 'applicazione $d: X \times X \rightarrow \R$ che soddisfa le seguenti proprietà : 
	\begin{enumerate}
		\item $d(x,y) \geq 0 $ per ogni $x,y \in X$ e vale $d(x,y)=0$ se e solo se x=y 
		\item $d(x,y)=d(y,x)$ per ogni $x,y \in X$ 
		\item $d(x,y) \leq d(x,z)+d(z,y) $ per ogni $x,y,z \in x$ (Disuguaglianza triangolare)
	\end{enumerate}
\end{definizione}
	\begin{es}
		Su un qualsiasi insieme X , la funzione 
		$$d: X \times X \rightarrow \R  \ \ \ d(x,y)= \begin{cases}
			0 \ \ \ x = y \\
			1 \ \ \ x \neq y 
		\end{cases}$$
		è una distanza 
	\end{es}
	\begin{definizione}
	Uno \textbf{spazio metrico} è una coppia $(X,d)$ , dove X è un insieme e d è una distanza su X 
	\end{definizione}
	\begin{definizione}
		Sia (X,d) uno spazio metrico. Il sottoinsieme 
		$$B(x,r) = \{y \in X | d(x,y)<r\}$$ viene detto palla paerta di centro x e raggio r 
	\end{definizione}
	\begin{definizione}[Topologia indotta da una distanza]Sia (X,d) uno spazio metrico. Nella topologia su X indotta dalla distanza d , un sottoinsieme $A \subset X$ è aperto se per ogni $x \in A $ esiste r > 0 tale che $B(x,r) \subset A $
	\end{definizione}
\begin{definizione}
	Siano $(X, d ) \ e \ (Y , \rho)$ due spazi metrici , sia f una funzione $f : X \rightarrow Y$ . f si dice Liptschiziana se esiste una costante $l \geq 0 $ tale che sia 
1	$$\rho(f(x),f(y)) \leq l \  d(x,y) \ \ \ \ \forall x , y \in X$$
\end{definizione}
\subsection{Ricoprimenti}
\begin{definizione}
	Un \textbf{ricoprimento} di un insieme X è una famiglia $\mathcal{A}$ di sottoinsieme tali che $X = \cup \{A \ | \ A \in \mathcal{A}\}$. diremo che il ricoprimento è finito se $\mathcal{A}$ è una famiglia finita : numerabile se $\mathcal{A}$ è una famiglia numerabile . \\
	Se $\mathcal{A} \ e \ \mathcal{B}$ sono ricoprimento di X se $\mathcal{A} \subset \mathcal{B}$ , allora dire che $\mathcal{A}$ è un \textbf{sottoricoprimento } di $\mathcal{B}$
  \end{definizione}
  \begin{definizione} Un ricoprimento $\A$ di uno spazio topologico X si dice : 
  \begin{itemize}
  	\item aperto se ogni $A \in \A$ è aperto 
  	\item chiuso se ogni $A \in \A$ è chiuso
  	\item localmente finito se per ogni punto $x \in X $ esiste un aperto $V \subset X$ tale che $x \in V  \ e \  V \cap A \neq \emptyset$ per al più un numero finito di $A \in \A$ 
  \end{itemize}
  \end{definizione}
  As esempio , ogni base della topologia è un ricoprimento aperto. 
  




\printbibliography[heading=bibintoc,title=Bibliografia]

	\end{document}