	\section{Teorema di Nagumo}
Utilizzerò la formulazione del teorema di Nagumo presentata nel libro "Viability Theory"\cite{VT}. Diamo ora alcune definizione necessarie 
\begin{definizione}
	Sia K un sottoinsieme di uno spazio vettoriale finito dimensionale (oppure di uno spazio normato) X . Diciamo che una funzione $x (\cdot) : [0,T] \rightarrow X$ è \textit{viable} in K su $[0,T]$ se 
	$$\forall t \in [0,T] \ , \ \ x(t) \in K$$ 
\end{definizione}
Consideriamo il seguente problema di cauchy 
\begin{equation}
	\label{diff:1}
	\begin{cases}
		\dot{x}(t)= f(x(t)) \ \ \ \forall t \in [0,T ] \\ 
		x(0) = x_0
	\end{cases}
\end{equation}
con $f : \Omega \subset_{\text{op}} X \rightarrow X$ 
\begin{definizione}
	Sia un sottoinsieme di $\Omega$. 
	Diciamo che K è\textit{ locally viable} sotto $f$ se per ogni condizione iniziale $x_0 \in K $ , esiste un $T > 0$ e una soluzione viable su $[0,T]$  per l'equazione differenziale \ref{diff:1} con condizione iniziale $x_0$\\
	K è (globalmente) viable sotto f se possiamo sempre prendere $T= \infty$ 
\end{definizione}
\begin{definizione}[Cono Tangente di Bouligand]
	Sia X uno spazio normato , K un sottoinsieme non vuoto di X e sia x un elemento di K . Il cono tangente a K in x è l'insieme
	$$T_K(x)= \{v \in X \ | \ \liminf_{h\rightarrow 0^+ }\frac{d_K(x+hv)}{h}=0\}$$
	con $d_k(x):= \inf_{y \in K } \norm{x-y}$
\end{definizione}
Una definizione alternativa utilizza le successioni : \\
v appartiene a $T_K(x)$ se e solo se esiste una successione $h_n >0$ $h_n \rightarrow 0^+$ e una successione $v_n \in X$ , $v_n \rightarrow v $ tale che 
$$\forall n \in \mathbb{N} \ , \ x+h_nv_n \in K $$
Il cono tangente a K nel punto x si può anche indicare con $T(K,x)$. 
\begin{figure}[h]
	\centering
	\includegraphics[width=0.5\linewidth]{immagini/ct}
	\caption{Due esempi di cono tangente}
	\label{fig:ct}
\end{figure}
Osserviamo che se x è un punto interno dell'insieme K allora $T_K(x)=X$. Quindi se l'insieme K è aperto ($K=K^0$) allora $\forall x \in K \ , \ T_k(x)=X$ . Saremmo quindi principalmente interessati a definire il cono tangente per punti apparteneti alla frontiera dell'insieme K . È importate sottolineare che può succedere che il cono tangente ad un punto di frontiera sia tutto lo spazio .
\begin{figure}[h]
	\centering
	\includegraphics[width=0.5\linewidth]{immagini/ct_all}
	\caption{Insieme K tale che $T_k(0)=X$}
	\label{fig:ct_all}
\end{figure}
\begin{lemma}
	Sia $x: [0,T] \rightarrow K $ una funzione differenziabile e viable , allora 
	$$\forall t \in [0,T) \ \ x'(t) \in T_K(x)$$
\end{lemma}
\begin{definizione}[Viability Domain]
	Sia  k un sottoinsieme di  $\Omega$. Diciamo che k è  un viability domain della mappa $f : \Omega \rightarrow X$ se 
	$$\forall x \in X  \ \ , \ \ f(x) \in T_K(x)$$
\end{definizione}
\begin{teo}{Nagumo}{}
	Suppiamo che il sottoinsieme K sia localmente compatto e che la funzione $f : K \rightarrow X$ sia continua . \\ Allora K è \textit{ localmente viable  } sotto f  se e solo K è un \textit{viability domain} della mappa f 
\end{teo}
\begin{es}
	Sia $K \subset \R \ , \ K =[-2,2]$ e sia $\dot{x}(t)=x(t)$. K è un intervallo chiuso e limitato , quindi compatto  ,  inoltre la funzione x è continua , quindi le ipotesi del teorema sono rispettate . Controlliamo se K è un viability domanin . Sappiamo che se $x \in (2,2)$ allora $T_K(x) = \R$ , analizziamo ora i due punti di frontiera : 
	\begin{itemize}
	\item  Se $x=2 \rightarrow f(x)=2 $ . Il cono tangente rappresenta le direzioni in cui posso muovermi senza uscire da K . In questo caso  posso unicamente muovermi verso sinistra quindi ottengo $T_K(2)= (-\infty , 0]$
	\item Facendo un ragionamento simile  a quello precedente ottengo $T_K(-2)= [0 , +\infty )$
	\end{itemize} 
	Vediamo che per entrambi i punti di frontiera otteniamo  $ f(x)\notin T_K(x) $ quindi   K non è viability domain per f . \\
	Prendiamo , ora l'intervallo $K=(-\infty , 0] $ , in questo caso il sottoinsieme non è compatto ma localmente compatto , poichè è un sottoinsieme di un insieme localmente compatto , lo spazio euclideo . \\Come prima se prendiamo $x \in (-\infty , 0 )$ otteniamo $T_k(x)=\R$ . Passiamo  ora a studiare il punto $x=0$ per cui otteniamo  $f(0)=0$ , invece il cono tangente nel punto zero è $T_k(0)= (-\infty , 0]$  . Possiamo osservare che $f(0) \in T_x(0)$ e quindi in generale K è un viabilty domain e quindi locally viable 
\end{es}
\begin{es}
	\label{es:1.2}
	Sia $K \subset \R^2 \ , \ K=\{(x,y)\in \R^2|x^2+y^2-5 \leq 0\} = \{(x,y)\in \R^2|g(x,y )\leq 0 \} $ e prendiamo $f(x,y)= ( x , y^2)$.K è un sottoinsieme  compatto e f è un campo vettoriale continuo . Sappiamo che per $x \in K^0 \rightarrow T_K(x)=\R^2$ . Occupiamoci  ora dei punti $x \in \partial K $ , quindi che appartengo alla curva di livello $f^{-1}(5)=\{(x,y)\in \R^2\ | \ x^2 +y^2 = 5 \}  $. In questo caso la curva è differenziale e possiamo definire il cono tangente attraverso rette tangenti infatti 
	$$T_k(x_0) =\{v \in X | g(x_0) + \grad g(x_0) \cdot v \leq 0 )\} $$
	Oppure possiamo utlizzare il fatto che il grandiente di g sia sempre perpeendicolare ad una curva di livello di g e che esso rappresenti la direzione di massima crescita 
	$$T_k(x_0)  = \{v \in X  | \grad g (x_0) \cdot v \leq 0  \} $$ . 
	Proviamo a calcolare la condizone di appartenenza al cono tangente 
\begin{align*}
		\grad{g(x)} f(x) &= (2x , 2y)( x , y^2)^T \\
		&= 2x^2 + 2y^3  \leq 0 \\
		y & \leq  - x^{\frac{2}{3}}
\end{align*}
	\begin{figure}[h]
		\centering
		\includegraphics[width=0.8\linewidth]{immagini/circonferenza}
		\caption{Esempio \ref{es:1.2}(a) }
		\label{fig:circonferenza} 
	\end{figure}\\
Infatti usando come punto $x_0= (0- , \sqrt{5})$
\begin{align*}
	f(x_0) & = (0 , 5) \\
	g(x_0) &= 0 \\
	\grad{g(x_0)} &= ( 0 , 2\sqrt{5})\\
	\grad{g(x_0)}  \cdot  f(x_0) &= 10 \sqrt{5} > 0 \rightarrow f(x_0) \notin T_k(x_0) 
\end{align*}
Abbiamo quindi dimostrato che K non è un viabilioty domain per f  , allora k non potrà essere locally viable sotto f . \\


	Proviamo ora a prendere lo stesso insieme K ma il campo vettoriale $f(x,y)= (-x , -y^3)$ . Vediamo se questo campo vettoriale rispetto la condizione di appartenenza al Cono tangente 
	\begin{align*}
		\grad{g(x)} f(x) &= (2x , 2y)(-x , -y^3) ^T \\
		&= -2x^2 -2y^4 \leq 0 \forall (x,y ) \in \R^2 
	\end{align*}
	Abbiamo quindi dimostrato che $f(x)\in T_k(x) \forall x \in K $ , quindi K è un viability domain per f e per il teorema d iNagumo K è locally viable sotto f  
		\begin{figure}[h]
		\centering
		\includegraphics[width=0.8\linewidth]{immagini/circonferenza_1}
		\caption{Esempio \ref{1.2}(b)  }
		\label{fig:circonferenza_1} 
	\end{figure}\\
	\end{es}
\begin{teo}{Viability}{}
	Consideriamo un sottoinsieme K di uno spazio finito dimensionale X e una mappa continua  $f : K \rightarrow X$. \\ Se K è un viability domain , allora per ogni condizione iniziale $x_0\in K $ esiste un T positivo e una soluzione viable su $[0,T]$ per l'equazione differenziale \ref{diff:1} con C.I $x_0$ tale che 
	$$\begin{cases}
		T= + \infty  \\ 
		T < +\infty \ \ \ \  e \ \ \ \  \limsup_{t\rightarrow T_-} \norm{x(t)}= \infty 
	\end{cases}$$
\end{teo}